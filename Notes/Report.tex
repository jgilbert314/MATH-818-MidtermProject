\documentclass[10pt,a4paper,fleqn]{article}
\usepackage[utf8]{inputenc}
\usepackage[T1]{fontenc}
\usepackage{amsmath}
\usepackage{amssymb}
\usepackage{graphicx}
\title{Math 818 \\ Report}
\author{Jason Gilbert}


\usepackage{float}
\usepackage{siunitx}
\setlength{\parindent}{0pt}



% General
\newcommand{\hamH}{\hat{H}}
% Oscillator
\newcommand{\hPhi}{\hat{\Phi}}
\newcommand{\hQ}{\hat{Q}}
% Transmon
\newcommand{\tranOp}{\hat{b}}
% Master equation
\newcommand{\bS}{\bar{n}_\kappa}
\newcommand{\dissOp}{\mathcal{D}}
\newcommand{\oscLad}{\hat{a}}
\newcommand{\phiCo}{\Phi_{\text{zpf}}}
\newcommand{\QCo}{Q_{\text{zpf}}}


\begin{document}
\maketitle
\newpage


\section{Important Equations}
\begin{table}[H]
	\centering
	\begin{tabular}{|c|l|l|}
		\hline
		Symbol & Meaning & Value \\
		\hline
		\#in & Inline math appearing near equation \# & \\
		\#der & Equation derived from equation \# & \\
		$ \hbar $ & Reduced Planck's constant & \SI{1.054571817e-34}{\joule \cdot \second} \\
		$ e $ & Charge of Electron & \SI{1.602e-19}{\coulomb} \\
		\hline
	\end{tabular}
\caption{Important Notation}
\label{tab:notation}
\end{table}


\subsection{Oscillator Equations} % p.4
Starting with the Hamiltonian for the oscillator (an LC circuit)
\begin{equation}\label{eq:oscHam}\tag{2}
	H_{LC} = \dfrac{Q^2}{2C} + \dfrac{1}{2}C\omega_r^2\Phi^2 \ ,
\end{equation}
the corresponding quantized obserables are
\begin{equation}\label{eq:oscOps}\tag{4}
	\hPhi = \phiCo(\oscLad^\dag + \oscLad) \ , \qquad
	\hQ = i\QCo(\oscLad^\dag - \oscLad) \ ,
\end{equation}
with coefficients
\begin{equation}\label{eq:oscCoeff}\tag{4in}
	\phiCo = \sqrt{\dfrac{\hbar Z_r}{2}} \ , \qquad
	\QCo = \sqrt{\dfrac{\hbar}{2Z_r}}
\end{equation}

In these terms the Hamiltonian is defined
\begin{equation}\label{eq:oscHamOp}\tag{5}
	\hamH_s = \hbar\omega_r\left(\oscLad^\dag\oscLad + \dfrac{1}{2}\right)
\end{equation}

Throughout this project, the following matrix representation of the ladder operators will be used
\begin{equation}\label{eq:ladMat}\tag{5ex}
	\oscLad_{nm} = \left\lbrace\begin{matrix}
		\sqrt{i} \quad \text{when}\ & n=m-1 \\
		0 \quad & \text{otherwise}
	\end{matrix}
	\right.
\end{equation}


\subsection{Transmon} % p.9-10
The Hamiltonian for the transmon is
\begin{equation}\label{eq:transHamil}\tag{27}
	\hamH = \sqrt{8E_C E_J}\tranOp^\dag \tranOp - \dfrac{E_C}{12}(\tranOp^\dag + \tranOp)^4
\end{equation}
where the corresponding ladder operators $ \tranOp $ and $ \tranOp^\dag $ can be related to $ \oscLad $ and $ \oscLad^\dag $ via Eq.(24-25) and
\begin{equation}\label{eq:tranOps}\tag{22in} % p.9
	\hat{n} = \dfrac{\hQ}{2e} \ , \qquad
	\hat{\varphi} = \dfrac{2e}{\hbar}\hPhi
\end{equation}
to obtain
\begin{equation}\label{eq:tranLad}\tag{25-26der}
	\tranOp = \dfrac{1}{2}\left( \dfrac{\hat{\varphi}}{\varphi_0} - \dfrac{\hat{n}}{n_0} \right) \ , \qquad
	\tranOp^\dag = \dfrac{1}{2}\left( \dfrac{\hat{\varphi}}{\varphi_0} + \dfrac{\hat{n}}{n_0} \right)
\end{equation}
where
\begin{equation}\label{eq:tranCoeff}\tag{25-26der-a}
	\varphi_0 = \left(\dfrac{2E_C}{E_J}\right)^{1/4} \ , \qquad
	n_0 = \dfrac{i}{2}\left( \dfrac{E_J}{2E_C} \right)^{1/4}
\end{equation}



\subsection{Transmons Coupled Via Resonator}
The Hamiltonian for a pair of transmons coupled by a resonator is given by
\begin{equation}\label{eq:resCoupleTrans}\tag{138}
	\hamH = \hamH_{q1} + \hamH_{q2} + \hbar\omega_r\oscLad^\dag\oscLad + \sum_{i=1}^{2}\hbar g_i(\oscLad^\dag\tranOp_i + \oscLad\tranOp_i^\dag)
\end{equation}


\subsection{Master Equation} % p.19
The master equation for the harmonic oscillator is given by
\begin{equation}\label{eq:masterEq}\tag{70}
	\dot{\rho} = -i[\hamH_s, \rho] + \kappa(\bS + 1)\dissOp[\oscLad]\rho + \kappa\bS\dissOp[\oscLad^\dag]\rho \ ,
\end{equation}
where the damping factor $ \kappa $ is defined according to
\[
	\kappa = 2\pi\lambda(\omega_r)^2 = Z_{tml}\omega_r^2\dfrac{C_\kappa^2}{C_r},
\]
and the dissipation operator is
\begin{equation}\label{eq:dissOp}\tag{71}
	\dissOp[A]B = ABA^\dag - \dfrac{1}{2}\lbrace A^\dag A, \ B \rbrace
\end{equation}
where $ A $ and $ B $ are operators, and $ \lbrace A, B \rbrace$ is the anti-commutator of $ A $ and $ B $.

The equivalent form for the transmon is given by
\begin{equation}\label{eq:masterEqTran}\tag{79} % p.20
	\dot{\rho} = -i[\hamH_q, \rho] + \gamma(\bar{n}_\gamma + 1)\dissOp[\tranOp]\rho + \gamma\bar{n}_\gamma\dissOp[\tranOp^\dag]\rho \ ,
\end{equation}
where $ \bar{n}_\gamma $ and $ \gamma $ are constants.



\subsection{Misc Parameters}

The equation for the qubit frequency is given by
\begin{equation}\label{eq:qubitFreq}\tag{27in} % p.10
	\omega_q = \sqrt{8E_C E_J} - E_C \ ,
\end{equation}
where $ E_J $ is the Josephson energy and $ E_C $ is the charging energy.

	
\end{document}